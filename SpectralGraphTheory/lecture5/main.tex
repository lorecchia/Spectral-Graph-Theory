\documentclass[11pt]{article}
\usepackage{graphicx}
\usepackage{amsmath,amsthm,amsfonts}
\usepackage{epsfig,graphics}
\usepackage{hyperref}
\usepackage{verbatim}
\usepackage{mathrsfs}
\usepackage{xcolor}

\setlength{\textheight}{8.5in}
\setlength{\evensidemargin}{0.0in}
\setlength{\oddsidemargin}{0.0in}
\setlength{\topmargin}{-0.5in}
\setlength{\textwidth}{6.5in}

\newtheorem{theorem}{Theorem}
\newtheorem*{theorem*}{Theorem}
\newtheorem{claim}{Claim}
\newtheorem*{claim*}{Claim}
\newtheorem{lemma}{Lemma}
\newtheorem*{lemma*}{Lemma}
\newtheorem{proposition}{Proposition}
\newtheorem*{proposition*}{proposition}
\newtheorem{exercise}{Exercise}
\newtheorem*{exercise*}{Exercise}
\newtheorem{corollary}{Corollary}
\theoremstyle{definition}
\newtheorem{definition}{Definition}
\newtheorem{fact}{Fact}
\newtheorem*{fact*}{Fact}



\newcommand{\handout}[6]{
   \renewcommand{\thepage}{#1-\arabic{page}}
   \noindent
   \begin{center}
      \vbox{
    \hbox to \textwidth { #2 \hfill #3 }
       \vspace{4mm}
       \hbox to \textwidth { {\Large \hfill #4  \hfill} }
       \vspace{2mm}
       \hbox to \textwidth { { #5 \hfill #6} }
      }
   \hrulefill
   \end{center}
   \vspace*{4mm}
}

%\newcommand{\lecture}[5]{\handout{#1}{#2}{#3}{#4}{#5}}


% Types of Variables
\newcommand{\bvar}[1]{\mathbf{#1}} % bold variable
\newcommand{\mvar}[1]{\bvar{#1}} % matrix variable
\newcommand{\vvar}[1]{\vec{#1}} % vector variable

% Domains
\newcommand{\R}{\mathbb{R}}
\newcommand{\Z}{\mathbb{Z}}
\newcommand{\redgevec}{\R^{E}}
\newcommand{\rvertvec}{\R^{V}}
\newcommand{\rPos}{\R^{+}}
\newcommand{\rNonNeg}{\R^{\geq 0}}

% Symbol for definitions
\newcommand{\defeq}{\stackrel{\mathrm{\scriptscriptstyle def}}{=}}

% Optimization
\DeclareMathOperator*{\argmin}{arg\,min}
\DeclareMathOperator{\Cone}{Cone}
\DeclareMathOperator{\Cut}{CUT}

% Types of Graphs
\newcommand{\nlap}{\mathscr{L}_G}
\newcommand{\pseudo}[1]{{#1}^\dagger}
\newcommand{\lapPseudo}{\pseudo{\lap}}
\newcommand{\adj}{A}
\newcommand{\incMatrix}{\mvar{B}}
\newcommand{\diag}{\operatorname{diag}}
\newcommand{\rMatrix}{\mvar{R}} % resistance matrix
\newcommand{\iMatrix}{\mvar{I}} % identity matrix
\newcommand{\Vol}{\textrm{Vol}}


% Vectors
\newcommand{\1}{\vec{1}}
\renewcommand{\dot}[1]{\langle {#1} \rangle}

%other
\DeclareMathOperator{\sgn}{sgn}


\usepackage[symbol]{footmisc}
\usepackage{color}
\renewcommand{\thefootnote}{\fnsymbol{footnote}}
\setlength\parindent{0pt}
\begin{document}

\handout{}{CS 591 O1: Iterative Methods for Graph Algorithms and Network Analysis}{Fall 2018}{Lecture 5: More about $\lambda_2$}{Instructor: Lorenzo Orecchia}{Scribe: Erasmo Tani}

\section*{Recap}
So far we have looked at:
\begin{enumerate}
    \item Role of the Laplacian,
    \item Continuous Random Walk,
    \item Heat Kernel: $e^{-t(I-W)}$ this is the solution to the differential equation:
    \[
        \frac{dp(t)}{dt} = -(I-W)p(t)
    \]
    this is known as the heat equation or diffusion process.
    We have seen how the convergence of this process depends on the second eigenvalue of the Laplapcian $\lambda_2$.:
    \[
        ||p(t) - \pi||_{D^{-1}}^2 \leq e^{-t\lambda_2}||p(0)- \pi||_{D^{-1}}^2
    \]
    \item We got some bounds on $\lambda_2$ for the cycle graph $C_n$ and we found: $\lambda_2 \leq O(\frac{1}{n^2})$ using a test vector to find x such that: $x^T1 = 0$:
    \[
        \lambda_2 \leq \frac{x^TLx}{x^TDx} = O\left(\frac{1}{n^2}\right)
    \]
    this characterization of $\lambda_2$ is useful to prove that for some starting conditions the process takes a long time to converge. However it is sometimes useful to try and prove that convergence is fast.
\end{enumerate}
    [...]
    
    \begin{lemma}
    $\lambda_2 =0 $ iff $G$ is disconected.
    \end{lemma}
    \begin{proof}
    Using the Rayleigh quotient characterization of the eigenvalues:
    \[
        \lambda_2 = \underset{x \perp 1}{\min}\,\frac{x^TLx}{x^TDx} 
    \]
    If the graph is disconnected then there exists at least two disconnected components, say $S$, and $\overline{S}$, we can then take $\mathbf{x}$ above to be:
    \[
        x_i = \begin{cases}\frac{1}{|S|} & \text{ if } i \in S\\
       - \frac{1}{|\overline{S}|}& \text{ if } i \in \overline{S}
        \end{cases}
    \]
    which gives:
    \[
        \lambda_2 \leq 0
    \]
    but we know:
    \[
        L \succeq 0
    \]
    and hence:
    \[
        \lambda_2 = 0.
    \]
    On the other hand, if $\lambda_2$ is zero, then there exists some eigenvalue $\mathbf{x} \perp 1$.  But then some entries of $\mathbf{x}$ are non-negative and some are negative. By looking at the quadratic form, it should be clear that there cannot be any edge connecting the set of vertices $i$ where the sign of $x_i$ non-negative, and those in which the sign is negative.
    \end{proof}
    ({\color{red} Spoiler Alert:} We will later see a robust version of this fact: if there is a small eigenvalue then there is a small cut. - Cheeger inequality).

    We then have the following generalization:
    \begin{proposition}
    $G$ has $k$ connected components if and only if $\lambda_k = 0$.
    \end{proposition}
    \begin{proof}
        The proof is somewhat similar to the above and makes use of the following characterization of the $k^{th}$ smallest eigenvalue, known as the Courant-Fischer theorem:
        \[
            \lambda_k = \underset{\substack{S \subseteq \R^n,\\ dim(s) = k}}{\min} \underset{x \in S}{\max} \frac{x^TLx}{x^TDx}
        \]
    \end{proof}
    \section*{Lower Bounds on $x^TLx$}
    \subsection*{Path graph}
    \begin{align}\label{eqt:path-inequality}
        x^TLx &= \sum_{i=1}^{n-1} (x_i - x_{i+1})^2 \geq \frac{(x_1 - x_n)^2}{n-1}
    \end{align}
    where the inequality follows by applying Cauchy-Schwartz with the $\vec{1}$ vector and the $n-1$-dimensional vector with entry $i$ equal to $(x_i-x_{i+1})$. $(\ref{eqt:path-inequality})$ is sometimes known as the \emph{path inequality}. In the above we have essentially compared the Laplacian of the path graph with the Laplacian of another graph. This is sometimes referred to as a \emph{graphic inequality}.
    
    \subsection*{Background: PSD Ordering}
    We write:
    \[
        \forall x: \hspace{0.5cm} x^TAx \geq x^TBx
    \]
    as:
    \[
        A \succeq B
    \]
    this is equivalent to:
    \[
        A- B \succeq 0
    \].
    When this is true, we have that the $k^{th}$ eigenvalue of $A$ is greater than the $k^{th}$ eigenvalue of $B$. Some strange things happen when looking at the PSD ordering, for instance:
    \[
        A \succeq B \not\Rightarrow A^2 \succeq B^2
    \]
    this is a consequence of non-commutativity.
    \section*{Graphic Inequalities}
    \begin{enumerate}
        \item $(n-1)L_{path} \succeq L_{1n}$,
        \item Question: $\lambda_2(L_{path}) \geq ?$ which graph should I compare this against? We will use the complete graph, the Laplacian of which is $L(K_v) = nI - 11^T$ and hence has eigenvalues $0, n, n, ..., n$.
        
    \end{enumerate}
    How does one interpret these graphic inequalities? It helps to think of the consequences that these inequalities have on random walk processes.
    \section*{Lowerbounds on $\lambda_2$}
    $(n-1)L_p \geq L_{1n}$ for any pairs of vertices $i,j$:
    $(j-1)L_P \geq L_{ij}$
    We then have:
    \[
        \sum_{i<j} (j-i)L_P \geq \sum_{i<j}L_{i,j} = L(K_v)
    \]
    which gives:
    \[
        O(n^3) \geq L(K_v)
    \]
    \[
        O(n^3)\lambda_2(L_{path}) \geq n
    \]
    \[
        O\left(\frac{1}{n^2}\right) \geq \lambda_2{(L_{path})} \geq \Omega\left(\frac{1}{n^2}\right)
    \]
    
    \section*{Complete Binary Tree}
    Just like the path the complete binary tree is acyclic and removing an edge disconnects the graph. However the $\lambda_2$ can be very different.
\end{document}
