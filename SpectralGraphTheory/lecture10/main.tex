\documentclass[11pt]{article}
\usepackage{graphicx}
\usepackage{amsmath,amsthm,amsfonts}
\usepackage{epsfig,graphics}
\usepackage{hyperref}
\usepackage{verbatim}
\usepackage{mathrsfs}
\usepackage{xcolor}

\setlength{\textheight}{8.5in}
\setlength{\evensidemargin}{0.0in}
\setlength{\oddsidemargin}{0.0in}
\setlength{\topmargin}{-0.5in}
\setlength{\textwidth}{6.5in}

\newtheorem{theorem}{Theorem}
\newtheorem*{theorem*}{Theorem}
\newtheorem{claim}{Claim}
\newtheorem*{claim*}{Claim}
\newtheorem{lemma}{Lemma}
\newtheorem*{lemma*}{Lemma}
\newtheorem{exercise}{Exercise}
\newtheorem*{exercise*}{Exercise}
\newtheorem{corollary}{Corollary}
\theoremstyle{definition}
\newtheorem{definition}{Definition}
\newtheorem{fact}{Fact}
\newtheorem*{fact*}{Fact}



\newcommand{\handout}[6]{
   \renewcommand{\thepage}{#1-\arabic{page}}
   \noindent
   \begin{center}
      \vbox{
    \hbox to \textwidth { #2 \hfill #3 }
       \vspace{4mm}
       \hbox to \textwidth { {\Large \hfill #4  \hfill} }
       \vspace{2mm}
       \hbox to \textwidth { { #5 \hfill #6} }
      }
   \hrulefill
   \end{center}
   \vspace*{4mm}
}

%\newcommand{\lecture}[5]{\handout{#1}{#2}{#3}{#4}{#5}}


% Types of Variables
\newcommand{\bvar}[1]{\mathbf{#1}} % bold variable
\newcommand{\mvar}[1]{\bvar{#1}} % matrix variable
\newcommand{\vvar}[1]{\vec{#1}} % vector variable

% Domains
\newcommand{\R}{\mathbb{R}}
\newcommand{\Z}{\mathbb{Z}}
\newcommand{\redgevec}{\R^{E}}
\newcommand{\rvertvec}{\R^{V}}
\newcommand{\rPos}{\R^{+}}
\newcommand{\rNonNeg}{\R^{\geq 0}}

% Symbol for definitions
\newcommand{\defeq}{\stackrel{\mathrm{\scriptscriptstyle def}}{=}}

% Optimization
\DeclareMathOperator*{\argmin}{arg\,min}
\DeclareMathOperator{\Cone}{Cone}
\DeclareMathOperator{\Cut}{CUT}

% Types of Graphs
\newcommand{\nlap}{\mathscr{L}_G}
\newcommand{\pseudo}[1]{{#1}^\dagger}
\newcommand{\lapPseudo}{\pseudo{\lap}}
\newcommand{\adj}{A}
\newcommand{\incMatrix}{\mvar{B}}
\newcommand{\diag}{\operatorname{diag}}
\newcommand{\rMatrix}{\mvar{R}} % resistance matrix
\newcommand{\iMatrix}{\mvar{I}} % identity matrix
\newcommand{\Vol}{\textrm{Vol}}


% Vectors
\newcommand{\1}{\vec{1}}
\renewcommand{\dot}[1]{\langle {#1} \rangle}

%other
\DeclareMathOperator{\sgn}{sgn}


\usepackage[symbol]{footmisc}
\usepackage{color}
\renewcommand{\thefootnote}{\fnsymbol{footnote}}
\newcommand{\norm}[1]{\left\lVert#1\right\rVert}

\setlength\parindent{0pt}
\begin{document}

\handout{}{CS 591 O1: Iterative Methods for Graph Algorithms and Network Analysis}{Fall 2018}{Lecture 10: Finishing Off Cheeger and Further Partitioning}{Instructor: Lorenzo Orecchia}{Scribe: Erasmo Tani}

\section*{Finishing Cheeger}
Recap from last time: We were giving a proof based on a metric point of view:
\begin{itemize}
    \item There is a characterization of conductance:
    \[
        \overline{\phi}:= \underset{S \subseteq V}{\min} \frac{|E(S,\overline{S})|}{|E(S,\overline{S})|} = \underset{d \in CUT_V}{\min} \, \frac{d(G)}{d(K_G)}
    \]
    \item $CUT_V:= $ metrics that are $\ell_1$-embeddable
\end{itemize}

We will follow the following plan:
\begin{itemize}
    \item Start with an eigenvector $v \perp \vec{1}$ s.t.: $\frac{x^TLx}{v^TL(K_G)v} = \lambda_{2,1}$ use $v$ to constrcut an $\ell_1$-metric $d$:
    \[
        d_{i,j}= || x_i - x_j||_1
    \]
    s.t.:
    \[
        \frac{d(G)}{d(K_G)}  \leq 2 \sqrt{2 \lambda_2}
    \]
\end{itemize}

We will find a small conductance cut by looking at sweep cuts corresponding to $\ell_1$-embeddings.

\begin{enumerate}
    \item Assume $\vec{x}$ is translated such that:
    \[
        \sum_{x_i \geq 0} d_i= \frac{1}{2}Vol(G)
    \]
    \item We let $y_i = \text{sign}(x_i) \cdot x_i^2$ (tensoring trick)
\end{enumerate}
We have:
\begin{align*}
    d(G) = \sum_{\{i,j\} \in E} |y_i - y_j| &= \sum_{E} |\text{sign}(x_i)\cdot x_i^2 - \text{sign}(x_j) \cdot x_j^2|\\
    &\leq \sum_E |x_i - x_j| (|x_i| + |x_j|)\\
    &\leq \sqrt{\sum_E (|x_i| + |x_j|)^2 \cdot \sum_E (x_i - x_j)^2}\\
    &\leq \sqrt{2 \sum_E (x_i^2 +x_j^2) x^TLx}\\
    &= \sqrt{2 x^TDx \cdot x^TLx}
\end{align*}
We can now look at the denominator:
\begin{align*}
    d(K_G) = \sum_{i < j } \frac{d_id_j}{Vol(G)} |y_i - y_j|
\end{align*}
We can now use the assumption in (1.) and we will only look at contributions that cross the ``zero line'':
\begin{align*}
    d(K_G) &= \sum_{i < j } \frac{d_id_j}{Vol(G)} |y_i - y_j|\\
    &\geq \sum_{x_i >0>x_j}\frac{d_id_j}{Vol(G)}\left(|y_i| +|y_j|\right)\\
    &\geq \sum_{x_i > 0} \frac{d_i \frac{Vol(G)}{2}}{Vol(G)} |y_i| + \sum_{x_j<0} \frac{d_j \frac{Vol(G)}{2}}{Vol(G)}|y_j|\\
    &= \frac{1}{2}\left(\sum_{i \in V} d_i |y_i|\right) = \frac{1}{2}\left(\sum d_i x_i^2\right) = \frac{1}{2}x^TDx
\end{align*}
Giving:
\[
    \frac{d(G)}{d(K_G)}\leq 2 \sqrt{2 \frac{x^TLx}{x^TDx}}\leq 2 \sqrt{2 \lambda_2}
\]

\section*{More Approaches to Graph Partitioning}
\subsection*{Graph Partitioning by Metric Relaxation}
Cheeger:
\[
    \min d(G)
\]
\[
    \text{s.t.}\hspace{2mm} d(K_G) = 1 \hspace{5mm} d \in CUT_V
\]
We will now use a linear program relaxation instead. In particular we will use a network flow linear program, a type of linear program that can be solved efficiently.
\[
    \min d(G)
\]
\[
    \text{s.t.}\hspace{2mm} d(K_G) = 1 \hspace{5mm} d \in \text{metric}_V
\]
this is known as the Leighton-Rao relaxation. In this class we will talk about the relaxation and we will postpone the discuss of the rounding to next time.
\begin{align*}
    \min & \sum_{\{i,j\} \in E} \delta_{i,j}\\
    \text{s.t.} & \sum_{i <j} \frac{d_id_j}{Vol(G)}\delta_{i,j} = 1\\
    &\forall i,j \forall p \in P_{i,j} \left[\delta_{i,j} \leq \sum_{e \in P}\delta_e\right]
\end{align*}

In the optimal solution, every $\delta_{i,j}$ is given by a path along $G$

\begin{align*}
    \min & \sum_{\{i,j\} \in E} \ell_{i,j}\\
    \text{s.t.} & \sum_{i <j} \frac{d_id_j}{Vol(G)}\delta_{i,j} = 1\\
    &\forall i,j \forall p \in P_{i,j} \left[\delta_{i,j} \leq \sum_{e \in P}\ell_e\right]
\end{align*}
While this primal problem might seem a little daunting due to the exponential number of constraints, it turns out its dual is a relatively simple flow problem:
\subsection*{Dual of the Leighton-Rao problem: Maxima Multicommodity Concurrent Flow}
Note that whenever your primal has some kind of triangle inequality constraint, its dual will be a flow problem of some kind.

\begin{align*}
    \max \hspace{2mm}&\hspace{2mm} \alpha \\
    \text{s.t.}\hspace{2mm} &\hspace{2mm}\forall e \in E: \sum _{e \in P} f_P \leq 1\\
    &\forall i,j  \sum_{p \in P_{i,j}}f_P \geq  \alpha \cdot \frac{d_id_j}{Vol(G)}
\end{align*}
in some sense you're trying to route $K_G$ into $G$ as a flow, but that is not always possible so you might have to scale things by a factor of $
\alpha$ which you want to maximize. Finding a feasible $\alpha$ immediately certifies the conductance is at least $\alpha$. We will see next time that LR does well on a path graph.
\end{document}
