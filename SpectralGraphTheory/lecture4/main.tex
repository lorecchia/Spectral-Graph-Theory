\documentclass[11pt]{article}
\usepackage{graphicx}
\usepackage{amsmath,amsthm,amsfonts}
\usepackage{epsfig,graphics}
\usepackage{hyperref}
\usepackage{verbatim}
\usepackage{mathrsfs}
\usepackage{xcolor}

\setlength{\textheight}{8.5in}
\setlength{\evensidemargin}{0.0in}
\setlength{\oddsidemargin}{0.0in}
\setlength{\topmargin}{-0.5in}
\setlength{\textwidth}{6.5in}

\newtheorem{theorem}{Theorem}
\newtheorem*{theorem*}{Theorem}
\newtheorem{claim}{Claim}
\newtheorem*{claim*}{Claim}
\newtheorem{lemma}{Lemma}
\newtheorem*{lemma*}{Lemma}
\newtheorem{exercise}{Exercise}
\newtheorem*{exercise*}{Exercise}
\newtheorem{corollary}{Corollary}
\theoremstyle{definition}
\newtheorem{definition}{Definition}
\newtheorem{fact}{Fact}
\newtheorem*{fact*}{Fact}



\newcommand{\handout}[6]{
   \renewcommand{\thepage}{#1-\arabic{page}}
   \noindent
   \begin{center}
      \vbox{
    \hbox to \textwidth { #2 \hfill #3 }
       \vspace{4mm}
       \hbox to \textwidth { {\Large \hfill #4  \hfill} }
       \vspace{2mm}
       \hbox to \textwidth { { #5 \hfill #6} }
      }
   \hrulefill
   \end{center}
   \vspace*{4mm}
}

%\newcommand{\lecture}[5]{\handout{#1}{#2}{#3}{#4}{#5}}


% Types of Variables
\newcommand{\bvar}[1]{\mathbf{#1}} % bold variable
\newcommand{\mvar}[1]{\bvar{#1}} % matrix variable
\newcommand{\vvar}[1]{\vec{#1}} % vector variable

% Domains
\newcommand{\R}{\mathbb{R}}
\newcommand{\Z}{\mathbb{Z}}
\newcommand{\redgevec}{\R^{E}}
\newcommand{\rvertvec}{\R^{V}}
\newcommand{\rPos}{\R^{+}}
\newcommand{\rNonNeg}{\R^{\geq 0}}

% Symbol for definitions
\newcommand{\defeq}{\stackrel{\mathrm{\scriptscriptstyle def}}{=}}

% Optimization
\DeclareMathOperator*{\argmin}{arg\,min}
\DeclareMathOperator{\Cone}{Cone}
\DeclareMathOperator{\Cut}{CUT}

% Types of Graphs
\newcommand{\nlap}{\mathscr{L}_G}
\newcommand{\pseudo}[1]{{#1}^\dagger}
\newcommand{\lapPseudo}{\pseudo{\lap}}
\newcommand{\adj}{A}
\newcommand{\incMatrix}{\mvar{B}}
\newcommand{\diag}{\operatorname{diag}}
\newcommand{\rMatrix}{\mvar{R}} % resistance matrix
\newcommand{\iMatrix}{\mvar{I}} % identity matrix
\newcommand{\Vol}{\textrm{Vol}}


% Vectors
\newcommand{\1}{\vec{1}}
\renewcommand{\dot}[1]{\langle {#1} \rangle}

%other
\DeclareMathOperator{\sgn}{sgn}


\usepackage[symbol]{footmisc}
\usepackage{color}
\renewcommand{\thefootnote}{\fnsymbol{footnote}}
\setlength\parindent{0pt}
\begin{document}

\handout{}{CS 591 O1: Iterative Methods for Graph Algorithms and Network Analysis}{Fall 2018}{Lecture 4: Continuos Time Random Walks and Spectral Gap}{Instructor: Lorenzo Orecchia}{Scribe: Erasmo Tani}

\section*{Continuous-Time Random Walk}
The continuous random walk is the process described by the differential equation:

\[
\frac{d p(t)}{dt} = -(I-W)p(t)
\]
Consider a $d$-regular graph, i.e. a graph in which every vertex $v \in V$ has degree $d_v = d$. Here $W= \frac{A}{d}$. Using the decomposition $(I-W) = U \Lambda U^T$ we get:
\[
    \frac{d p(t)}{dt} = -(U \Lambda U^T)p(t)
\]
(since in this case $(I-W)$ is symmetric).


By applying a change of basis we can write:
\[
    \frac{d U^T p(t)}{dt} = -( \Lambda U^T)p(t)
\]
Which is equivalent to having, for $i=1, \cdots, n$:
\[
    \left(\frac{dU^Tp(t)}{dt}\right)_i = - \lambda_i\left(U^Tp(t)\right)_i
\]

by solving each first order ODE we get:
\[
    \left(U^Tp(t)\right)_i = \left((e^{-\lambda_i}U^Tp(0)\right)_i
\]
\textcolor{blue}{edit:
\[
    \left(U^Tp(t)\right)_i = \left((e^{-\lambda_it}U^Tp(0)\right)_i
\]
}

so that:
\[
    U^Tp(t) = \text{diag}\left(e^{-\lambda_1}, ... , e^{-\lambda_n}\right)U^Tp(0)
\]
\textcolor{blue}{edit:
\[
    U^Tp(t) = \text{diag}\left(e^{-\lambda_1t}, ... , e^{-\lambda_nt}\right)U^Tp(0)
\]
}

and hence:
\begin{equation}\label{eqn:markov-chain1}
    p(t) = U^T\text{diag}\left(e^{-\lambda_1}, ... , e^{-\lambda_n}\right)U^Tp(0)
\end{equation}
\textcolor{blue}{edit:
\begin{equation}\label{eqn:markov-chain1}
    p(t) = U\text{diag}\left(e^{-\lambda_1t}, ... , e^{-\lambda_nt}\right)U^Tp(0)
\end{equation}
}
    
\section*{Intermezzo: Matrix Exponential}
For the sake of convenience, we introduce here the notion of matrix exponential.
\begin{definition}[Matrix Exponential]\label{def:matrix-exponential}
Given a symmetric matrix $M = U\Lambda U^T$, where $\Lambda =\text{diag}(\lambda_1, ... , \lambda_n)$ the \emph{matrix exponential} of $M$ is the matrix:
\[
    e^{M} = U\text{diag}(e^{\lambda_1, ... , \lambda_n})U^T
\]
\textcolor{blue}{edit:
\[
    e^{M} = U\text{diag}(e^{\lambda_1}, ... , e^{\lambda_n})U^T
\]}

so that $e^M$ has the same eigenvector as $M$ but eigenvalues related by: $\lambda_i(e^M) = e^{\lambda_i(M)}$.
\end{definition}

Note that:
\begin{align*}
    e^Y &= U^T\text{diag}_i\left(\sum_{j=0}^{\infty}\frac{\lambda_i^j}{j!}\right)U\\
    &= \sum_{j=0}^{\infty}\frac{1}{j!} U^T\text{diag}_i(\lambda_i^j)U\\
    &= \sum_{j=0}^{\infty}\frac{Y^j}{j!}
\end{align*}

so that we recover an matrix analogue of the familiar Taylor expansion:
\[
\boxed{e^Y = \sum_{j=0}^{\infty}\frac{Y^j}{j!}}
\]

We also have the following property, which we state without a proof:
\begin{proposition}
Given any two symmetric matrices $A$ and $B$ we have:
\[
    AB = BA \hspace{1cm} \Rightarrow \hspace{1cm} e^{A}e^{B} = e^{A+B}
\]
\end{proposition}

Note that in general $e^{A+B} \neq e^Ae^B$.

\section*{Back to the Markov Chain}
In light of Definition \ref{def:matrix-exponential} we can rewrite Equation $(\ref{eqn:markov-chain1})$ as:

\[
    p(t) = e^{-\left(I - \frac{A}{d}\right)}p(0)
\]

\begin{align*}
    \frac{dD^{-1/2}p(t)}{dt} &= D^{-1/2}(I-AD^{-1})p(t)\\
    &= D^{-1/2}(I-AD^{-1})p(t)\\
    &= (I-D^{-1/2}AD^{-1/2})D^{-1/2}p(t)
\end{align*}

\textcolor{blue}{edit: (maybe note that starting here the graph is not necessarily regular?)
\begin{align*}
    \frac{dD^{-1/2}p(t)}{dt} &= -D^{-1/2}(I-AD^{-1})p(t)\\
    &= - (I-D^{-1/2}AD^{-1/2})D^{-1/2}p(t)
\end{align*}
}


\begin{definition}[Normalized Laplacian]
Given a graph $G$ its \emph{normalized Laplacian} is the matrix:
\[
    \mathcal{L} = (I- D^{-1/2}AD^{-1/2}) = D^{-1/2}L D^{-1/2}
\]
\end{definition}

we then have:
\begin{align*}
    D^{-1/2}p(t) &= e^{-t \mathcal{L}}D^{-1/2} p(0)    
\end{align*}
so that
\begin{align*}
    p(t) &= D^{1/2}e^{-t\mathcal{L}}D^{-1/2}p(0)
\end{align*}
so that the derivative of the distance to the stationary distribution is given by:
\begin{align*}
    \frac{d}{dt}||p - \pi ||^2_{D^{-1}} &= \frac{d}{dt} ||D^{1/2} e^{-t\mathcal{L}}D^{1/2} (p^{(0)} - \pi) ||^2_{D^{-1}}\\
    &= -(p(0) - \pi)^T D^{-1/2}e^{-t\mathcal{L}}\mathcal{L}e^{-t\mathcal{L}}D^{-1/2}(p(0) - \pi)\\
    &= -(D^{-1}(p(t) - \pi))^{-1}L(D^{-1}(p(t) - \pi))
\end{align*}

In order to evaluate our progress we consider the ratio:
\[
    \frac{\frac{d}{dt}||p(t) - \pi||^2_{D^{-1}}}{||p(t) - \pi||^2_{D^{-1}}}
\]

we have:
\begin{align*}
    \frac{\frac{d}{dt}||p(t) - \pi||^2_{D^{-1}}}{||p(t) - \pi||^2_{D^{-1}}} &= - \frac{(p(t) - \pi)^TD^{-1/2} \mathcal{L} D^{-1/2}(p(t) - \pi)}{(p(t) - \pi)^TD^{-1/2}I D^{-1/2}(p(t) - \pi)}\\
    &\leq \underset{D^{1/2}y^T1 =0}{\min}\frac{y^T\mathcal{L} y}{y^Ty}
\end{align*}
\textcolor{blue}{edit:
\begin{align*}
    \frac{\frac{d}{dt}||p(t) - \pi||^2_{D^{-1}}}{||p(t) - \pi||^2_{D^{-1}}} &= - \frac{(p(t) - \pi)^TD^{-1/2} \mathcal{L} D^{-1/2}(p(t) - \pi)}{(p(t) - \pi)^TD^{-1/2}I D^{-1/2}(p(t) - \pi)}\\
    &\leq - \underset{D^{1/2}y^T1 =0}{\min}\frac{y^T\mathcal{L} y}{y^Ty}
\end{align*}
}


\begin{theorem}
\[
    ||p(t) - \pi||^2_{D^{-1}} \leq e^{-t\lambda_2}(p(0) - \pi)^2_{D^{-1}}
\]
\end{theorem}

\begin{definition}[Spectral Gap]
The spectral Gap is the quantity:
\[
\lambda_2(\mathcal{L}) = \underset{D^{1/2}y^T1 =0}{\min}\frac{y^T\mathcal{L} y}{y^Ty} 
\]

this equals:
\[
    \underset{x^TD1}{\min} \frac{x^TLx}{x^TDx}
\]
\end{definition}
\textcolor{blue}{edit: 
\[
    \underset{x^TD1=0}{\min} \frac{x^TLx}{x^TDx}
\]}



\section*{Examples of Spectral Gaps}

\subsection*{Spectral Gap of Complete Graphs}
The complete graph $K_n$ has Laplacian matrix:
\[
    L(K_n) = nI - 11^T
\]
\[
    D(K_n) = (n-1)\cdot I
\]
And hence the spectral gap is given by:
\[
    \frac{n}{n-1} = 1 + \frac{1}{n-1}
\]
which is the largest possible.

\subsection*{Spectral Gap of n-Cycle}
The spectral gap of an $n$-cycle is $O\left(\frac{1}{n^2}\right)$
\end{document}
