 \section{Introduction to the course}

Welcome to CS 591 O1! At the beginning of the lecture we went through the logistics of the course. Information about that can be found at the course webpage:
\begin{center}
\url{http://cs-people.bu.edu/orecchia/CS591fa16/course.html}
\end{center}

\subsection{Subject Matter}

In this course, we will look at a new ``continuous'' perspective on graph problems and graph algorithms, which differs from the traditional combinatorial approach. Our perspective will be rooted in ideas from convex optimization and linear algebra. This larger viewpoint will allow us to understand graph objects in a number of new ways, often by leveraging geometric and analytical tools. For instance, we will describe graphs as matrices and draw different results by interpreting these matrices as linear operators, quadratic functions or as descriptions of random walk processes over the graph. 

We will focus in particular on using our continuous view to construct fast simple algorithms for important graph problems. In the process, we will also review and develop a toolkit of iterative methods from convex optimization.

The course will be roughly divided in three parts:
\begin{enumerate}
\item {\bf Spectral Graph Theory}: Graphs as Linear Operators, Random Walks and their Convergence, Applications to Clustering and Distributed Algorithms;
\item {\bf Graphs as Electrical Circuits:}  Electrical Flows and their interpretations, Laplacian linear systems, Power Method, Preconditioning;
\item {\bf Iterative Methods in Convex Optimization:} Smooth and Non-Smooth Optimization, Smoothing, Coordinate Descent, Applications to Graph Algorithms.
\end{enumerate}
