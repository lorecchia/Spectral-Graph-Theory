\documentclass[11pt]{article}
\usepackage{graphicx}
\usepackage{amsmath,amsthm,amsfonts}
\usepackage{epsfig,graphics}
\usepackage{hyperref}
\usepackage{verbatim}
\usepackage{mathrsfs}
\usepackage{xcolor}

\setlength{\textheight}{8.5in}
\setlength{\evensidemargin}{0.0in}
\setlength{\oddsidemargin}{0.0in}
\setlength{\topmargin}{-0.5in}
\setlength{\textwidth}{6.5in}

\newtheorem{theorem}{Theorem}
\newtheorem*{theorem*}{Theorem}
\newtheorem{claim}{Claim}
\newtheorem*{claim*}{Claim}
\newtheorem{lemma}{Lemma}
\newtheorem*{lemma*}{Lemma}
\newtheorem{proposition}{Proposition}
\newtheorem*{proposition*}{proposition}
\newtheorem{exercise}{Exercise}
\newtheorem*{exercise*}{Exercise}
\newtheorem{corollary}{Corollary}
\theoremstyle{definition}
\newtheorem{definition}{Definition}
\newtheorem{fact}{Fact}
\newtheorem*{fact*}{Fact}



\newcommand{\handout}[6]{
   \renewcommand{\thepage}{#1-\arabic{page}}
   \noindent
   \begin{center}
      \vbox{
    \hbox to \textwidth { #2 \hfill #3 }
       \vspace{4mm}
       \hbox to \textwidth { {\Large \hfill #4  \hfill} }
       \vspace{2mm}
       \hbox to \textwidth { { #5 \hfill #6} }
      }
   \hrulefill
   \end{center}
   \vspace*{4mm}
}

%\newcommand{\lecture}[5]{\handout{#1}{#2}{#3}{#4}{#5}}


% Types of Variables
\newcommand{\bvar}[1]{\mathbf{#1}} % bold variable
\newcommand{\mvar}[1]{\bvar{#1}} % matrix variable
\newcommand{\vvar}[1]{\vec{#1}} % vector variable

% Domains
\newcommand{\R}{\mathbb{R}}
\newcommand{\Z}{\mathbb{Z}}
\newcommand{\redgevec}{\R^{E}}
\newcommand{\rvertvec}{\R^{V}}
\newcommand{\rPos}{\R^{+}}
\newcommand{\rNonNeg}{\R^{\geq 0}}

% Symbol for definitions
\newcommand{\defeq}{\stackrel{\mathrm{\scriptscriptstyle def}}{=}}

% Optimization
\DeclareMathOperator*{\argmin}{arg\,min}
\DeclareMathOperator{\Cone}{Cone}
\DeclareMathOperator{\Cut}{CUT}

% Types of Graphs
\newcommand{\nlap}{\mathscr{L}_G}
\newcommand{\pseudo}[1]{{#1}^\dagger}
\newcommand{\lapPseudo}{\pseudo{\lap}}
\newcommand{\adj}{A}
\newcommand{\incMatrix}{\mvar{B}}
\newcommand{\diag}{\operatorname{diag}}
\newcommand{\rMatrix}{\mvar{R}} % resistance matrix
\newcommand{\iMatrix}{\mvar{I}} % identity matrix
\newcommand{\Vol}{\textrm{Vol}}


% Vectors
\newcommand{\1}{\vec{1}}
\renewcommand{\dot}[1]{\langle {#1} \rangle}

%other
\DeclareMathOperator{\sgn}{sgn}


\usepackage[symbol]{footmisc}
\usepackage{color}
\renewcommand{\thefootnote}{\fnsymbol{footnote}}
\newcommand{\norm}[1]{\left\lVert#1\right\rVert}
\newcommand{\bu}{\mathbf{u}}

\setlength\parindent{0pt}
\begin{document}

\handout{}{CS 591 O1: Iterative Methods for Graph Algorithms and Network Analysis}{Fall 2018}{Lecture 7: More on Sparsification }{Instructor: Lorenzo Orecchia}{Scribe: Erasmo Tani}

\section*{Sparsification}
We reduced the problem to the setting in which we have a collection of vectors in $n$ dimension $\bu_i \in \R^n$ and :
\begin{equation}\label{eqt:first}
    \sum_{i}\bu_i\bu_i^T = I
\end{equation}
here we want to select and weight some of these elements. So that:
\[
    \sum_{i=1}^m w_i \bu_i\bu_i^T \approx I
\]
and the vector $\mathbf{w} \in \R^m$ is sparse. The strategy we will use is oblivious sampling at each iteration $t=1, ... ,T$ we sample a vector $i_t$ w.p.: $p_{i_t} = p_i$ (independent of $t$):
\[
    w_{i_t} = \frac{1}{p_{i_t}}
\]
Note that Equation $(\ref{eqt:first})$ implies $\sum_i ||\bu_i||^2 = n$

\begin{theorem}
If we choose $p_i \propto ||u_i||^2$ (importance sampling) then, after $T = O(\frac{n \log n}{\epsilon^2})$ samples:
\[
    \bigg\| \sum_{t=1}^T w_{i_t} \bu_{i_t}\bu_{i_t}^T - I\bigg\| \leq \epsilon
\]
with constant probability (the norm above is the spectral norm w.r.t. $\ell_2$).
\end{theorem}
To prove this, we will make use of the Matrix Chernoff Bound. The $O(\frac{n\log n}{\epsilon^2})$ is optimal for oblivious sampling (probability doesn't change with iteration counter), but not in general. Adaptive sampling allows for $O(\frac{n}{\epsilon^2})$. Consider the example of a dumbbell graph. The problem with oblivious sampling is that the probability of the middle edge to be sampled is roughly $\frac{1}{n}$ which gives the $O(n\log n)$ (this has interesting connections with the balls into bins process). We will now prove Theorem 1:
\begin{proof}
\begin{claim}
$\mathbb{E}[w_{i_t}\bu_{i_t}\bu_{i_t}^T] = \sum_{i=1}^m p_i\frac{1}{p_{i}}\bu_{i_t}\bu_{i_t}^T = \sum \bu_{i_t}\bu_{i_t}^T = I$
\end{claim}

\vspace{5mm}
Now, we will introduce the Laplacian transform potential function.
Let $A_t = \sum_{i=1}^t w_{i_s} u_{i_s}u_{i_s}^T$. Then:

\[
\lambda_{\max}(A_t) \approx \frac{1}{\eta} \log Tr(e^{\eta A_t}) = \frac{1}{\eta} \log (\sum e^{\eta \lambda_i(A_t)})
\]
we define the following potential function:
\[
    \phi_t = \mathbb{E}[Tr e^{\eta A_t}].
\]
How does the potential change?
\begin{align*}
    \mathbb{E}[\phi_{(t+1)}]&= \mathbb{E}[Tr e^{\eta A_{t+1}}] = \mathbb{E}[Tr e^{\eta(A_t + A_{t+1})}]
\end{align*}
Where the expecttion is conditional on al randomness up to time $t$. From here we can apply the Golden-Thomson inequality:
\begin{theorem}[Golden-Thomson Inequality]
For symmetric matrices $X$ and $Y$:
\[
    Tr(e^{x+y}) \leq Tr(e^xe^y).
\]
\end{theorem}
We then get:
\begin{align*}
    \mathbb{E}[\phi_{(t+1)}]&= \mathbb{E}[Tr e^{\eta A_{t+1}}] = \mathbb{E}[Tr e^{\eta(A_t + A_{t+1})}] \leq Tr(e^{\eta A_t}\cdot\mathbb{E}_{t+1}[e^{\eta \Delta_{t+1}}]) = e^{\eta A_t}\bullet\mathbb{E}_{t+1}[e^{\eta \Delta_{t+1}}] 
\end{align*}
Where we define:
\[
    A \bullet B:= Tr(A^TB)    
\]
We then have:
\[
    e^{\eta \Delta} \preceq I + (e^{\eta ||\Delta||} -1)\frac{\Delta}{||\Delta||}
\]
Note that we have:
\[
    0 \leq\Delta \leq ||\Delta||\vec{1}
\]
\[
    e^{\eta x} \leq 1 + (e^{\eta ||\Delta||} -1)\frac{X}{||\Delta||}
\]

(Note: We also have that if $A \succeq 0$, $B \preceq C$ then:
\[
    A \bullet B \leq A \bullet C
\]
). We then have:
\begin{align*}
    \mathbb{E}[\phi_{(t+1)}]\leq Tr(e^{\eta A_t}) + e^{\eta A_t} \boldsymbol{\cdot} \frac{\Delta}{||\Delta||} \cdot(e^{\eta ||\Delta||} - 1)
\end{align*}
and hence:
\[
    \frac{\mathbb{E}[\phi_{t+1}]}{\phi_{t+1}} = 1+\frac{e^{\eta A_t}}{Tr(e^{\eta A_t})}\bullet\mathbb{E}\left[ \frac{\Delta}{||\Delta||}(e^{\eta||\Delta||} -1)\right] 
\]
but also:
\[
    ||\Delta_{t+1}||=\norm{\frac{n}{||u||^2}u u^T} = \frac{n}{||u||^2}||u||^2 n
\]
so that:
\begin{align*}
    \frac{\mathbb{E}[\phi_{t+1}]}{\phi_{t+1}}&\leq 1 +\left(\frac{e^{\eta n }-1}{n}\right)
\end{align*}

\begin{align*}
    \frac{\mathbb{E}[\phi_T]}{\phi_0} \leq \left(1 + \frac{e^{\eta n}-1}{n}\right)^T\phi_0 \leq \left(1 + \frac{e^{\eta n}-1}{n}\right)^Tn.
\end{align*}
Now:
\begin{align*}
 \text{Pr}\left[\lambda_{\max}(A_T) > T(1+\varepsilon)\right] &\leq \text{Pr}\left[\phi_{T} > e^{\eta T(1+ \varepsilon)}\right]\\
 & \leq \frac{\left(1+\frac{e^{\eta n}-1}{n}\right)^Tn}{e^{\eta T(1+\varepsilon)}}\\
 &= e^{\frac{\varepsilon}{2}\frac{T}{n}(-\varepsilon) + \log n}
\end{align*}
So that $T = c\frac{n \log n}{\varepsilon^2}$.\\
\end{proof}
We want:
\[
    p_e = \frac{\norm{L^{-\frac{1}{2}}\chi_e}^2}{n} = \frac{\chi_e^TL^{-1}\chi_e}{n}     
\]
We will see that we can compute all the $p_e$s in time $O(m \log n)$.
\section*{}


\end{document}
