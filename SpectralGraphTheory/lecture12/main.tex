\documentclass[11pt]{article}
\usepackage{graphicx}
\usepackage{amsmath,amsthm,amsfonts}
\usepackage{epsfig,graphics}
\usepackage{hyperref}
\usepackage{verbatim}
\usepackage{mathrsfs}
\usepackage{xcolor}

\setlength{\textheight}{8.5in}
\setlength{\evensidemargin}{0.0in}
\setlength{\oddsidemargin}{0.0in}
\setlength{\topmargin}{-0.5in}
\setlength{\textwidth}{6.5in}

\newtheorem{theorem}{Theorem}
\newtheorem*{theorem*}{Theorem}
\newtheorem{claim}{Claim}
\newtheorem*{claim*}{Claim}
\newtheorem{lemma}{Lemma}
\newtheorem*{lemma*}{Lemma}
\newtheorem{exercise}{Exercise}
\newtheorem*{exercise*}{Exercise}
\newtheorem{corollary}{Corollary}
\theoremstyle{definition}
\newtheorem{definition}{Definition}
\newtheorem{fact}{Fact}
\newtheorem*{fact*}{Fact}



\newcommand{\handout}[6]{
   \renewcommand{\thepage}{#1-\arabic{page}}
   \noindent
   \begin{center}
      \vbox{
    \hbox to \textwidth { #2 \hfill #3 }
       \vspace{4mm}
       \hbox to \textwidth { {\Large \hfill #4  \hfill} }
       \vspace{2mm}
       \hbox to \textwidth { { #5 \hfill #6} }
      }
   \hrulefill
   \end{center}
   \vspace*{4mm}
}

%\newcommand{\lecture}[5]{\handout{#1}{#2}{#3}{#4}{#5}}


% Types of Variables
\newcommand{\bvar}[1]{\mathbf{#1}} % bold variable
\newcommand{\mvar}[1]{\bvar{#1}} % matrix variable
\newcommand{\vvar}[1]{\vec{#1}} % vector variable

% Domains
\newcommand{\R}{\mathbb{R}}
\newcommand{\Z}{\mathbb{Z}}
\newcommand{\redgevec}{\R^{E}}
\newcommand{\rvertvec}{\R^{V}}
\newcommand{\rPos}{\R^{+}}
\newcommand{\rNonNeg}{\R^{\geq 0}}

% Symbol for definitions
\newcommand{\defeq}{\stackrel{\mathrm{\scriptscriptstyle def}}{=}}

% Optimization
\DeclareMathOperator*{\argmin}{arg\,min}
\DeclareMathOperator{\Cone}{Cone}
\DeclareMathOperator{\Cut}{CUT}

% Types of Graphs
\newcommand{\nlap}{\mathscr{L}_G}
\newcommand{\pseudo}[1]{{#1}^\dagger}
\newcommand{\lapPseudo}{\pseudo{\lap}}
\newcommand{\adj}{A}
\newcommand{\incMatrix}{\mvar{B}}
\newcommand{\diag}{\operatorname{diag}}
\newcommand{\rMatrix}{\mvar{R}} % resistance matrix
\newcommand{\iMatrix}{\mvar{I}} % identity matrix
\newcommand{\Vol}{\textrm{Vol}}


% Vectors
\newcommand{\1}{\vec{1}}
\renewcommand{\dot}[1]{\langle {#1} \rangle}

%other
\DeclareMathOperator{\sgn}{sgn}


\usepackage[symbol]{footmisc}
\usepackage{color}
\renewcommand{\thefootnote}{\fnsymbol{footnote}}
\newcommand{\norm}[1]{\left\lVert#1\right\rVert}

\setlength\parindent{0pt}
\begin{document}

\handout{}{CS 591 O1: Iterative Methods for Graph Algorithms and Network Analysis}{Fall 2018}{Lecture 11: Flow-Based Partitioning and Rounding }{Instructor: Lorenzo Orecchia}{Scribe: Erasmo Tani}
\section*{Rounding of the Leighton-Rao Relaxation}
distance over $G$ by lenght over the edges.
\[
    W = \sum_{e\in E} \ell_e
\]
\[
    \sum_{v_1, v_2\in V} d_{i,j} \geq 1
\]
Te above says that the average distance is about $\frac{1}{\binom{n}{2}}$. We want to round this to an $\ell_1$-metric. We will conside two cases: an easy case and a hard case.
\begin{itemize}
    \item[(Easy case)] There exists a component $T\subseteq V$ with radius $\frac{1}{2n^2}$ such that:
    \[
        |T| \geq \frac{2}{3}
    \]
    The intuition is that there is a subset of points that has very low variance and hence the remaining points must lie far away from them. Formally we have the following claim:\\
    \textbf{Claim:} in this case we have:
    \[
        \sum_{u \in V\backslash T} d(u,T)\geq \frac{1}{2n}
    \]
    \begin{proof}
        We have: $\forall u,v$:
        \[
            d(u,v)\leq d(u,T) + d(v,T) + \frac{1}{n^2}
        \]
        since $1/n^2$ is the diameter of $T$. But then:
        \[
            1 \leq \sum_{\{u,v\}} d_{u,v} = \frac{1}{2}\sum_{(u,v)\in V\times V} \left( d(T,u) + d(T,v) + \frac{1}{n^2}\right) < n \cdot \sum_{w \in V \setminus T}d(T,w) + \frac{1}{2}
        \]
        giving us the desired result.
    \end{proof}
    \textbf{Goal:} come up with an $\ell_1$-embedding $\vec{y}_i$ with $i\in V$ such that:
    \begin{enumerate}
        \item \[\frac{\sum ||y_i - y_j ||}{\sum_{\{u,v\}}||y_u - y_v||} \leq 3W
        \]
        We will use a Frechet-type embedding:
        \[
            y_i = d(T,i)
        \]
        One nice property of these embeddings is that, for all $i,j$:
        \[
            |y_i - y_j| \leq d_{ij}
        \]
        so that the embedding is always a contraction. And we then have:
        \[
            \sum_{\{i,j\}\in E}|y_i - y_j|\leq \sum_{\{i,j\}\in E} d_{i,j} = W
        \]
        \begin{align*}
        \sum_{\{u,v\}} |y_u - y_v| &\geq  \sum_{u \in V \setminus T}\sum_{v \in T}|y_u - y_v| = |T| \sum_{u \in V \setminus T}y_u = |T|\cdot \frac{1}{2n}\geq \frac{1}{3}
        \end{align*}
    \end{enumerate}
    \item[(Hard Case)]There exists no component $T\subseteq V$ with radius $\frac{1}{2n^2}$ such that:
    \[
        |T| \geq \frac{2}{3}
    \]
    Here, the following structure lemma comes in handy :
    \begin{lemma}[Structure Lemma]
    Assuming the above setup, for any $\Delta >0$ we can partition $G$ into components of radius $r \leq \Delta$ such that the number of edges connecting the different components is less than or equal:
    \[
        \frac{4W\log n}{\Delta}
    \]
    \end{lemma}
    Now, for our proof to work we will also assign a volume of $W / n$ to each vertex, so that the total volume (over edges and vertices together) is $2W$.
    We will give a constructive proof. Consider the following algorithm:
    \begin{enumerate}
        \item Pick a vertex $v$ and grow a ball in this pipe system. We define the following two quantities:
        \[
            B(v,r) := \{ u \in V\mid d_{uv}\leq r\}
        \]
        \[
            V(s,r):= \frac{W}{n}+ \sum_{\{u,v\} \subseteq B(s,r)} c_{uv}\ell_{u,v} + \sum_{\{u,v\} \in \delta(B(s,r))}c(u,v)(r- d_{us})
        \]
        Note that you only count the volume of a vertex if it is at the root. We can also talk about the \emph{capacity} of the ball:
        \[
            C(s,r) = \sum_{\{u,v\}\in \delta(B(s,r))\cap E} c_{uv}
        \]
        We then have:
        \[
            \frac{dV(s,r)}{dr} = C(s,r)
        \]
        whenever the derivative is defined. (In some points we will instead have a left derivative and a right derivative.)
        We will now track:
        \[
            \ln V(s, \Delta) - \ln V(s,0) = \ln \frac{V(s, \Delta)}{V(s,0)} = \int_{0}^1 \frac{1}{V(s,r)}\cdot \frac{\partial V(s,r)}{\partial r} dr = \int_{0}^{1} \frac{C(s,r)}{V(s,r)} dr
        \]
        \[
            \ln n = \ln \frac{W}{W/n} \geq \ln \frac{V(s,\delta)}{V(s,0)}\geq \Delta \cdot \underset{r \in [0,\Delta]}{\min}\frac{C(s,r)}{V(s,r)}
        \]
        You can find $B(s,r_S)$ such that:
        \[
            \frac{C(s,r_s)}{V(s,r_s)}\leq \frac{\log n}{\Delta}
        \]
        \[
        C(s,r_s) \leq \frac{\log n}{\Delta}\cdot V(s,r(s))
        \]
        We may now remove this ball and reiterate.
        \[
            \sum C(s, r_S) \leq \frac{\log n}{\Delta}
        \].
        So we can find a cut $(\overline{S},S)$ where $1/3n \leq |S| \leq 2/3n$
        \[
            \alpha_1(S, \overline{S}) = \frac{E(S, \overline{S})}{|S||\overline{S}} \leq \frac{4Wn^2 \log n}{n^2/9} = O(W \log n)
        \]
        An alternative proof of this can be done by constructing an $\ell_1$ embedding and showing that under the assumption of case $2$ we can find two balanced sets that are at least $\delta$-separated.
    \end{enumerate}
\end{itemize}
\end{document}
