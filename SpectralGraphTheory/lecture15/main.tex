\documentclass[11pt]{article}
\usepackage{graphicx}
\usepackage{amsmath,amsthm,amsfonts}
\usepackage{epsfig,graphics}
\usepackage{hyperref}
\usepackage{verbatim}
\usepackage{mathrsfs}
\usepackage{xcolor}

\setlength{\textheight}{8.5in}
\setlength{\evensidemargin}{0.0in}
\setlength{\oddsidemargin}{0.0in}
\setlength{\topmargin}{-0.5in}
\setlength{\textwidth}{6.5in}

\newtheorem{theorem}{Theorem}
\newtheorem*{theorem*}{Theorem}
\newtheorem{claim}{Claim}
\newtheorem*{claim*}{Claim}
\newtheorem{lemma}{Lemma}
\newtheorem*{lemma*}{Lemma}
\newtheorem{exercise}{Exercise}
\newtheorem*{exercise*}{Exercise}
\newtheorem{corollary}{Corollary}
\theoremstyle{definition}
\newtheorem{definition}{Definition}
\newtheorem{fact}{Fact}
\newtheorem*{fact*}{Fact}



\newcommand{\handout}[6]{
   \renewcommand{\thepage}{#1-\arabic{page}}
   \noindent
   \begin{center}
      \vbox{
    \hbox to \textwidth { #2 \hfill #3 }
       \vspace{4mm}
       \hbox to \textwidth { {\Large \hfill #4  \hfill} }
       \vspace{2mm}
       \hbox to \textwidth { { #5 \hfill #6} }
      }
   \hrulefill
   \end{center}
   \vspace*{4mm}
}

%\newcommand{\lecture}[5]{\handout{#1}{#2}{#3}{#4}{#5}}


% Types of Variables
\newcommand{\bvar}[1]{\mathbf{#1}} % bold variable
\newcommand{\mvar}[1]{\bvar{#1}} % matrix variable
\newcommand{\vvar}[1]{\vec{#1}} % vector variable

% Domains
\newcommand{\R}{\mathbb{R}}
\newcommand{\Z}{\mathbb{Z}}
\newcommand{\redgevec}{\R^{E}}
\newcommand{\rvertvec}{\R^{V}}
\newcommand{\rPos}{\R^{+}}
\newcommand{\rNonNeg}{\R^{\geq 0}}

% Symbol for definitions
\newcommand{\defeq}{\stackrel{\mathrm{\scriptscriptstyle def}}{=}}

% Optimization
\DeclareMathOperator*{\argmin}{arg\,min}
\DeclareMathOperator{\Cone}{Cone}
\DeclareMathOperator{\Cut}{CUT}

% Types of Graphs
\newcommand{\nlap}{\mathscr{L}_G}
\newcommand{\pseudo}[1]{{#1}^\dagger}
\newcommand{\lapPseudo}{\pseudo{\lap}}
\newcommand{\adj}{A}
\newcommand{\incMatrix}{\mvar{B}}
\newcommand{\diag}{\operatorname{diag}}
\newcommand{\rMatrix}{\mvar{R}} % resistance matrix
\newcommand{\iMatrix}{\mvar{I}} % identity matrix
\newcommand{\Vol}{\textrm{Vol}}


% Vectors
\newcommand{\1}{\vec{1}}
\renewcommand{\dot}[1]{\langle {#1} \rangle}

%other
\DeclareMathOperator{\sgn}{sgn}


\usepackage[symbol]{footmisc}
\usepackage{color}
\renewcommand{\thefootnote}{\fnsymbol{footnote}}
\newcommand{\norm}[1]{\left\lVert#1\right\rVert}

\setlength\parindent{0pt}
\begin{document}

\handout{}{CS 591 O1: Iterative Methods for Graph Algorithms and Network Analysis}{Fall 2018}{Lecture 15: More on SDP and MAXCUT}{Scribe: Erasmo Tani}

\section*{Last Time...}
MAXCUT: Given  $G=(V,E)$ we want to find $(S,\overline{S})$ so that $|E(S,\overline{S})|$ is maximized. Recall that there is a simple solution that gives a 1/2-approximation problem. We found that linear programming didn't beat this approximation guarantees. Spectral methods gave a 9/8 integrality gap.
\subsection*{Vector Embedding}
Relax: $x_i \in \R \to x_i \in \R^n$ for each vertex $v$. The SDP:

\begin{align*}
    \max \;& \;\frac{1}{4}\sum_{ij \in E} ||v_i - v_j||^2\\
    \text{s.t.}\;& \;\forall i\; ||v_i||^2
\end{align*}

for any cut $S$ we can consider the solution:
\[
    v_i^{(S)} = \begin{cases}1 &if i \in S\\
    -1 &if i \not\in S
    \end{cases}
\]
we then have:
\[
    \text{value}(v^{(S)}) = |E(S,\overline{S})|.
\]

Note that this is kind of similar to a spectral relaxation except here we have $n$ constraints. We have that the spectral case looks like:
\begin{align*}
    \max\; &\; \frac{1}{4}\sum ||v_i - v_j||\\
    s.t.\;&\;\sum d_i ||v_i||^2 = 2|E|.
\end{align*}


\section*{Goemans-Williamson Guarantees}
The given spectral relaxation is called the Goemans-Williamson relaxation. Today we willl prove that the Geomans-Williamson relaxation can be rounded to a $0.87856$-approximation to MAXCUT.

\subsection*{Rounding}
Pick a direction $s$ uniformly at random and look at the inner product of the solution vectors with $s$. We will assign $i\in S$ iff $\langle v_i , s\rangle \geq 0$.\\

\textbf{Proof of the Guarantee}Take a pair i,j then:
\begin{align*}
    \text{Pr}\left(ij \text{ is cut by rounding}\right) = \frac{\theta_{ij}}{\pi}
\end{align*}

where $\theta_{i,j}$ is the angle between $v_i$ and $v_j$. Note that one may only consider the projection of $s$ onto the 2-dimensional subspace spanned by $v_i$ and $v_j$. But recall:

\begin{align*}
    \theta_{ij} = \text{arccos}\langle v_i,v_j\rangle
\end{align*}

but then:
\begin{align*}
    \mathbb{E}\left[|E(S,\overline{S})|\right] = \sum_{\{i,j\}\in E} \text{arccos}(\langle v_i,v_j\rangle)
\end{align*}

We will compare the above expectation with the solution to the SDP that is an upperbound to the value of MAXCUT$(G)$.
\begin{align*}
    SDP = \frac{1}{4}\sum_{\{i,j\}\in E} ||v_i - v_j||^2 = \frac{1}{2}\sum_{\{i,j\}\in E}(1 - \langle v_i,v_j\rangle)
\end{align*}
Consider:
\begin{align*}
    \frac{\mathbb{E}\left[E(S,\overline{S})\right]}{SDP} = \frac{2}{\pi} \cdot \frac{\sum_{\{i,j\}\in E}\text{arccos }\langle v_i,v_j\rangle}{\sum_{\{i,j\}\in E}1-\langle v_i,v_j\rangle}
\end{align*}
We have:
\begin{align*}
    \underset{-1\leq x \leq 1}{\min}\;\frac{2}{\pi} \cdot \frac{\text{arccos}x}{1-x} \geq 0.878...
\end{align*}
where the minimum is attained at $x = -0.67$. There are also matching integrality gaps between these.

\section*{Partitioning: Minimum Conductance}
Recall:
\begin{align*}
    \underset{S \subseteq V}{\min}\; \frac{|E( S,\overline{S})|}{Vol(S)\cdot Vol(\overline{S})}\cdot Vol(G)
\end{align*}
Cheeger:
\[
    \lambda_2 \leq \overline{\phi}_G \leq 2\sqrt{2\lambda_2}
\]
Spectral Relaxation:
\begin{align*}
    \min \sum_{\{i,j\}} || v_i - v_j||^2\\
    s.t. \sum_{i < j} \frac{d_i d_j}{Vol(G)}||v_i - v_j||^2 = 1
\end{align*}

Note that this formulation might partition in the graph in two sets of very different volume.
\subsection*{Balanced Minimum Conductance}
Divide-and-conquer: we would to get a partition where a relatively large fraction of the volume lies on each side. We define the $b$-balanced minimum conductance:
\begin{align*}
    \overline{\phi}_{G,b} = \underset{S \subseteq V}{\min}\; \overline{\phi}(S)\\
    Vol(S) \geq b\cdot Vol(G)
\end{align*}
\textbf{Side Note: Integral solution to spectral relaxation}:
\begin{align*}
    (S,\overline{S})
\end{align*}
we may consider the vector:
\begin{align*}
    v_{i}^{(S)} = \begin{cases} \sqrt{\frac{Vol(G)}{Vol(S)Vol(\overline{S})}} & \text{ if } i \in S\\
    0 & \text{ if }  i \in \overline{S}
    \end{cases}
\end{align*}

Suppose now that we want to rule out cuts that are not balanced, i.e. we want to impose the constraint that:
\[
    \frac{Vol(S)}{Vol(G)} > b    
\]
or equivalently:
\[
    1- \frac{Vol(S)}{Vol(G)} = \frac{Vol(\overline{S})}{Vol(G)} < 1-b
\]
in order to make things balanced we might add a constraint:
\[
    ||v_i||^2 \leq \sqrt{Vol(G)} \cdot \frac{1}{b}
\]
Alternatively, we may think of the complete graph $K_G$ as the sum of a bunch of stars:
\[
    k_G = \frac{1}{2}\sum_{i \in V} d_i \cdot S_i^G
\]
where $S_i$ is a star graph centered at $i$. We may want to change the above blance constraint into a translation-invariant one:
\[
    d_{i}\sum_{j \in V}\frac{d_{i,j}}{Vol(G)}||v_i - v_j||^2
\]
where the above is the Laplacian quadratic form of $d_i\cdot S_i$.
\subsection*{SDP Relaxation}
\begin{align*}
    \min \;&\; L \cdot X\\
    \text{s.t.}\;&\;L(K_G) \cdot X = 1\\
    &\forall i: \; L(S_i)\cdot X \leq  \frac{Vol(G)}{b}
\end{align*}
or, equivalently:
\begin{align*}
    \min \;&\; L \cdot X\\
    \text{s.t.}\;&\;L(K_G) \cdot X = Vol(G)\\
    &\forall i: \; d_i\cdot L(S_i)\cdot X \leq  d_i \frac{1}{b}
\end{align*}
and the dual gives:
\begin{align*}
    \max \;&\; \alpha - \sum \frac{d_i\beta_i }{b}\\
    \text{s.t.}\;&\; L + \sum d_i \beta_i L(S_i) \succeq \alpha L(K_G)
\end{align*}
\end{document}
