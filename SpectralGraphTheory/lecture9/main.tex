\documentclass[11pt]{article}
\usepackage{graphicx}
\usepackage{amsmath,amsthm,amsfonts}
\usepackage{epsfig,graphics}
\usepackage{hyperref}
\usepackage{verbatim}
\usepackage{mathrsfs}
\usepackage{xcolor}

\setlength{\textheight}{8.5in}
\setlength{\evensidemargin}{0.0in}
\setlength{\oddsidemargin}{0.0in}
\setlength{\topmargin}{-0.5in}
\setlength{\textwidth}{6.5in}

\newtheorem{theorem}{Theorem}
\newtheorem*{theorem*}{Theorem}
\newtheorem{claim}{Claim}
\newtheorem*{claim*}{Claim}
\newtheorem{lemma}{Lemma}
\newtheorem*{lemma*}{Lemma}
\newtheorem{proposition}{Proposition}
\newtheorem*{proposition*}{proposition}
\newtheorem{exercise}{Exercise}
\newtheorem*{exercise*}{Exercise}
\newtheorem{corollary}{Corollary}
\theoremstyle{definition}
\newtheorem{definition}{Definition}
\newtheorem{fact}{Fact}
\newtheorem*{fact*}{Fact}



\newcommand{\handout}[6]{
   \renewcommand{\thepage}{#1-\arabic{page}}
   \noindent
   \begin{center}
      \vbox{
    \hbox to \textwidth { #2 \hfill #3 }
       \vspace{4mm}
       \hbox to \textwidth { {\Large \hfill #4  \hfill} }
       \vspace{2mm}
       \hbox to \textwidth { { #5 \hfill #6} }
      }
   \hrulefill
   \end{center}
   \vspace*{4mm}
}

%\newcommand{\lecture}[5]{\handout{#1}{#2}{#3}{#4}{#5}}


% Types of Variables
\newcommand{\bvar}[1]{\mathbf{#1}} % bold variable
\newcommand{\mvar}[1]{\bvar{#1}} % matrix variable
\newcommand{\vvar}[1]{\vec{#1}} % vector variable

% Domains
\newcommand{\R}{\mathbb{R}}
\newcommand{\Z}{\mathbb{Z}}
\newcommand{\redgevec}{\R^{E}}
\newcommand{\rvertvec}{\R^{V}}
\newcommand{\rPos}{\R^{+}}
\newcommand{\rNonNeg}{\R^{\geq 0}}

% Symbol for definitions
\newcommand{\defeq}{\stackrel{\mathrm{\scriptscriptstyle def}}{=}}

% Optimization
\DeclareMathOperator*{\argmin}{arg\,min}
\DeclareMathOperator{\Cone}{Cone}
\DeclareMathOperator{\Cut}{CUT}

% Types of Graphs
\newcommand{\nlap}{\mathscr{L}_G}
\newcommand{\pseudo}[1]{{#1}^\dagger}
\newcommand{\lapPseudo}{\pseudo{\lap}}
\newcommand{\adj}{A}
\newcommand{\incMatrix}{\mvar{B}}
\newcommand{\diag}{\operatorname{diag}}
\newcommand{\rMatrix}{\mvar{R}} % resistance matrix
\newcommand{\iMatrix}{\mvar{I}} % identity matrix
\newcommand{\Vol}{\textrm{Vol}}


% Vectors
\newcommand{\1}{\vec{1}}
\renewcommand{\dot}[1]{\langle {#1} \rangle}

%other
\DeclareMathOperator{\sgn}{sgn}


\usepackage[symbol]{footmisc}
\usepackage{color}
\renewcommand{\thefootnote}{\fnsymbol{footnote}}
\newcommand{\norm}[1]{\left\lVert#1\right\rVert}

\setlength\parindent{0pt}
\begin{document}

\handout{}{CS 591 O1: Iterative Methods for Graph Algorithms and Network Analysis}{Fall 2018}{Lecture 9: Metrics, Cuts, Conductance and all That Jazz }{Instructor: Lorenzo Orecchia}{Scribe: Erasmo Tani}

\section*{Introduction}
Recall:
\begin{itemize}
    \item (Symmetrized) Conductance: $\overline{\phi}(S) := \frac{|E(S,\overline{S}|}{Vol(S)Vol(\overline{S})} \cdot Vol(V)$,
    \item $\overline{\phi} = \underset{S \subseteq V}{\min} \overline{\phi}(S)= \underset{S \subseteq V}{\min}\frac{|E(S,\overline{S})|}{|E_{K_G}(S,\overline{S})|} = \underset{S \subseteq V}{\min}\frac{\vec{1}^T_S L_G\vec{1_S}}{|E_{K_G}(S,\overline{S})|} \geq \underset{x\in \R^n
    }{\min}\frac{x^TL_Gx}{x^TL_{K_G}x} =\lambda_2$ 
    \item if $\lambda_2$ is large, all cuts have large conductance.
\end{itemize}

\section*{Metrics}
\begin{definition}[Metric]
A metric can be seen as a matrix $d \in \R^{n \times n}$ satisfying:
\begin{enumerate}
    \item $\forall i$: $d_{ii} =0$
    \item $\forall i,j$: $d_{ij} =d_{ji}$
    \item $\forall i,j,k$:  $d_{ij} \leq d_{ik} + d_{kj}$
\end{enumerate}
\end{definition}
\subsection*{Examples}
\begin{enumerate}
    \item $L_2$-distance over $\R^n$,
    \item $L_2^2$-distance is \emph{not} in general a metric.
\end{enumerate}

\begin{definition}[Semimetric]
A semimetric is a metric which does not necessarily satisfy the triangle inequality.
\end{definition}
Given a metric $d$ on a graph $G$, we can obtain a notion of volume on $G$:
\[
    d(G):= \sum_{\{i,j\}\in E} w_{i,j} d_{i,j}
\]
\section*{Cut Metrics}
\begin{definition}[Cut Metric]
Given a set $S \subseteq V$, the cut metric $\delta^{(S)}$ is:
\[
\delta_{i,j}^{(S)}:=\begin{cases}0 &\text{ if }i,j\text{ are on the same side of the cut}\\
1 &\text{ otherwise}
\end{cases} 
\]
\end{definition}
\section*{Relation to Conductance}
\[
    \delta^{(S)}(G) = |E_G(S,\overline{S})|  
\]
\[
    \delta^{(S)}(K_G) = \frac{Vol(S)Vol(\overline{S})}{Vol(V)}
\]
\[
    \overline{\phi}(S) = \frac{\delta^{(S)}(G)}{\delta^{(S)}(K_G)}
\]
\[
    \overline{\phi}_G= \underset{\delta \in \text{cut metric}}{\min}\frac{\delta(G)}{\delta(K_G)}
\]
the above is a more general version of the ratio:
\[
    \underset{S \subseteq V}{\min}\;\frac{\vec{1}_S^TL\vec{1}_S}{\vec{1}_S^TL(K_G)\vec{1}_S}    
\]
\section*{Spectral Gap}
\[
\lambda_2  =\min \;\frac{\sum_{\{i,j\} \in E}w_{i,j}(x_i - x_j)^2}{\sum_{i,j\in V}\frac{d_id_j}{Vol(G)}(x_i - x_j)^2}
\]
this kind of looks like a ratio of volume.
\begin{definition}
A semimetric $g$ is $\ell_2^2$-\emph{embeddable} if there exists an embedding $v_i \in R^{\ell}$ (for some $\ell\in \mathbb{N}$) s.t.:
\[
   \forall i,j\in V g_{i,j}=||v_i - v_j||_2^2 
\]
\end{definition}

\begin{lemma}
\[
    \lambda_2: = \underset{g \; \ell_2^2\text{-embeddable}}{\min}\frac{g(G)}{g(K_G)}
\]
\end{lemma}
\begin{proof}
Note:
\[
    \boxed{\frac{\sum s_i}{\sum b_i} \geq \underset{i}{\min} \frac{a_i}{b_i}}
\]
The above can easily be proved by rewriting the LHS as:
\[
    \sum_{i} \frac{b_i}{\sum_i b_i}\cdot \frac{a_i}{b_i}
\]
We then have:
\begin{align*}
    \underset{g \in \ell_2^2}{\min} \; \frac{g(G)}{g(K_G)} &= \frac{\sum_{\{i,j\} \in E} w_{i,j}||v)i - v_j||^2}{\sum_{i,j\in V}\frac{d_id_j}{Vol(V)}||v_i - v_j||^2}\\
    &= \frac{\sum_{u=1}^k\sum_{\{i,j\} \in E}w_{i,j}(v_i - v_j)^2_u}{\sum_{u=1}^k\sum_{i,j \in V}\frac{d_id_j}{Vol(V)}||v_i - v_j||^2_u}\\
    &\geq \underset{1 \leq u \leq k}{\min} \frac{\sum_{\{i,j\} \in E}w_{i,j}(v_i - v_j)^2_u}{\sum_{i,j \in V}\frac{d_id_j}{Vol(V)}||v_i - v_j||^2_u}\\
    &\geq \lambda_2
\end{align*}
\end{proof}
Note: Cut metrics are $\ell_2^2$-embeddable.\\ 

The space of cut metrics is discrete, but we will consider its convex hull.
\section*{Cut Cone}
\begin{definition}
[Cone]
A set $A \in \R^k$ is a \emph{cone} if for all $x\in A$ and $\alpha >0$ we have $\alpha x\in A$.
\end{definition}
\begin{definition}[Convex Cone]
The convex cone generated by a set of points $B= \{v_i \in \R^k\}$ is the set:
\[
    Cone (B):= \left\{x \in \R^K: x = \sum \alpha_i v_i,\hspace{2mm} \vec{\alpha} >0 \right\}
\]
\end{definition}
\begin{definition}[Cut Cone]
The \emph{cut cone} $CUT_v$ is:
\[
    Cone(\{\delta^{(S)}\}_{S\subseteq V})
\]
\end{definition}
We will show $CUT_V =$metrics that embeddable in $\ell_1$.

\begin{definition}[$\ell_1$-embeddability]
A metric $d$ over $V$ is $\ell_1$-embeddable if there exists an embedding $\{v_i \in \R^k\}_{i\in V}$ s.t:
\[
    d_{i,j} =||v_i - v_j||_1
\]
\end{definition}
\begin{theorem}
The set of $\ell_1$-embeddable metrics over $V$ is $CUT_V$.
\end{theorem}
\begin{proof}
We will show in two parts:
\begin{enumerate}
    \item Any cut metric is $\ell_1$-embeddable
    \item Any $\ell_1$-embeddable metric is a conic combination of cut merics
\end{enumerate}
We make use of the Sweep Cut:
\[
    S_t= \{1, ... ,t\}
\]
we will have:
\[
    d = \sum|v_{t+1}- v_t|\cdot \delta^{(S_t)}
\]
\end{proof}
\end{document}
